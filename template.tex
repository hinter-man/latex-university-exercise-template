\documentclass[12pt]{article}
%\usepackage[ngerman]{babel}
\usepackage[english]{babel}
\usepackage{cmbright}
\usepackage[utf8]{inputenc}
\usepackage{fullpage}
\usepackage[T1]{fontenc}
\usepackage{graphicx}
\graphicspath{ {./img/} }
\usepackage{color}
\usepackage{pdfpages}
\usepackage[]{parskip}
\usepackage{tabularx}
\usepackage{mathabx}
\usepackage{hyperref}
\usepackage{float}

\usepackage[svgnames]{xcolor}
  \definecolor{diffstart}{named}{Grey}
  \definecolor{diffincl}{named}{Green}
  \definecolor{diffrem}{named}{OrangeRed}
  \definecolor{keywordstyle}{named}{Blue}
  \definecolor{numberstyle}{named}{Black}
  \definecolor{stringstyle}{named}{Red}
  \definecolor{commentstyle}{named}{Green}
  

\usepackage{listings}
% language to show differences
\lstdefinelanguage{diff}{
  frame=tblr,
  breaklines=true,
  basicstyle=\ttfamily\small,
  morecomment=[f][\color{diffstart}]{@@},
  morecomment=[f][\color{diffincl}]{+\ },
  morecomment=[f][\color{diffrem}]{-\ },
}

\lstset{
  language=C++, %Java, ...
  commentstyle=\color{commentstyle},
  keywordstyle=\color{keywordstyle},
  numberstyle=\tiny\color{numberstyle},
  stringstyle=\color{stringstyle},
  basicstyle=\footnotesize\ttfamily,
  breakatwhitespace=false,         
  breaklines=true,                                     
  keepspaces=true,                 
  numbers=left,                                      
  showspaces=false,                
  showstringspaces=false,
  showtabs=false,
  frame=tblr,
  caption=\lstname,
  frame=single
}


\title{Titel} % title of the page
\author{Manuel Hinterreiter\\S1910454015} 

\begin{document}
%\includepdf[pages=-]{../yourpdf.pdf}
\maketitle

\tableofcontents
\newpage
\section{Abschnitt eins}
\begin{lstlisting}[caption=Class1.cs]
public Class1()
{
    ...
}
\end{lstlisting}

\begin{lstlisting}[language=diff, caption=Diff.cs]
@@ -1,3 +1,5 @@
language=diff
-       hilfreich um Unterschiede zwischen zwei Files anzuzeigen
-       am besten auf Kommandozeilenebene
\ No newline at end of file
+       hilfreich um Unterschiede zwischen 2 Files anzuzeigen
+       am besten auf Kommandozeilenebene
+       git diff file1 file2
+       kopieren und einfuegen
\end{lstlisting}

\section{Abschnitt zwei}
\begin{table}[h!]
  \centering
  \begin{tabular}{|p{2.5cm}||p{4cm}|p{4cm}|}
    \hline
     \textbf{Spalte 1} & \textbf{Spalte 2} & \textbf{Spalte 3} \\
     \hline
     &&\\
     \hline
   \end{tabular}
   \caption{Beschreibung}
   \label{tab:table1}
\end{table}

\end{document}